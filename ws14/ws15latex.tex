


\documentclass[11pt,letterpaper]{article}

% Load some basic packages that are useful to have
% and that should be part of any LaTeX installation.
%
% be able to include figures
\usepackage{graphicx}
% get nice colors
\usepackage{xcolor}

% change default font to Palatino (looks nicer!)
\usepackage[latin1]{inputenc}
\usepackage{mathpazo}
\usepackage[T1]{fontenc}
% load some useful math symbols/fonts
\usepackage{latexsym,amsfonts,amsmath,amssymb}

% comfort package to easily set margins
\usepackage[top=1in, bottom=1in, left=1in, right=1in]{geometry}
\usepackage{hyperref}
\usepackage[all]{hypcap}
% control some spacings
%
% spacing after a paragraph\begin{figure}[bth]
\setlength{\parskip}{.15cm}
% indentation at the top of a new paragraph
\setlength{\parindent}{0.0cm}

\begin{document}

\begin{center}
\Large
Ay190 -- Worksheet 14\\
David Vartanyan\\
Date: \today
\end{center}

\section{}

I use advect\_prep template from ws11. I account for upwind velocity and fill in the template.

Not 
much
more
to
s
a
y
.

Here's a figure ~\ref{fig:1}.

We see a shock ~150s in this frame where material piles up at $x \approx 50$.

\begin{figure}[bth]
\centering
\includegraphics[width=0.7\textwidth]{ws14.png}
\caption{Advection Shock}
\label{fig:1}
\end{figure}

\end{document}
