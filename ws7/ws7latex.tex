\documentclass[11pt,letterpaper]{article}

% Load some basic packages that are useful to have
% and that should be part of any LaTeX installation.
%
% be able to include figures
\usepackage{graphicx}
% get nice colors
\usepackage{xcolor}

% change default font to Palatino (looks nicer!)
\usepackage[latin1]{inputenc}
\usepackage{mathpazo}
\usepackage[T1]{fontenc}
% load some useful math symbols/fonts
\usepackage{latexsym,amsfonts,amsmath,amssymb}

% comfort package to easily set margins
\usepackage[top=1in, bottom=1in, left=1in, right=1in]{geometry}
\usepackage{hyperref}
\usepackage[all]{hypcap}
% control some spacings
%
% spacing after a paragraph\begin{figure}[bth]
\setlength{\parskip}{.15cm}
% indentation at the top of a new paragraph
\setlength{\parindent}{0.0cm}

\begin{document}

\begin{center}
\Large
Ay190 -- Worksheet 07\\
David Vartanyan\\
Date: \today
\end{center}

\section{Exercise 1}


For small sample size $N$, error decreases with sample size following the $N^{-1/2}$ convergence for MC methods. However, we see for larger sample size $N$, the error becomes oscillatory. Roundoff error probably dominates here.

See Figure ~\ref{fig:1}.

SystemRandom uses system processes to generate random numbers instead of the seed algorithm using in the previous part. 

See Figure ~\ref{fig:2}

\section{Exercise 2}

We see that in a group $23$ people, the probability for a birthday among exactly two people to be shared is greater than $0.5$. We run this for a varying amount of runs. Increasing the runs(the sample sizes) returns expected convergence.

To compute this analytically: let $p_{dif}(n)$ be the probability for everyone in a group of $n$ people to have a different birthday.

Then, 

\begin{equation}
p_{dif}(n)=1\times(1-\frac{1}{365})\times(1-\frac{2}{365})...
\end{equation}

The probability for at least $2$ people to share a birthday is then $p(n)=1-p_{dif}(n)$.

If we simplify our expression for $p_{dif}$, we get

\begin{equation}
p(n)=1-\frac{365!}{365^{n}(365-n)!}
\end{equation}

\begin{figure}[bth]
\centering
\includegraphics[width=0.7\textwidth]{ws7a.png}
\caption{Random.Seed Error.}
\label{fig:1}
\end{figure}

\begin{figure}[bth]
\centering
\includegraphics[width=0.7\textwidth]{ws7b.png}
\caption{System.Random Error.}
\label{fig:2}
\end{figure}

\end{document}
