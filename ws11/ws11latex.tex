
\documentclass[11pt,letterpaper]{article}

% Load some basic packages that are useful to have
% and that should be part of any LaTeX installation.
%
% be able to include figures
\usepackage{graphicx}
% get nice colors
\usepackage{xcolor}

% change default font to Palatino (looks nicer!)
\usepackage[latin1]{inputenc}
\usepackage{mathpazo}
\usepackage[T1]{fontenc}
% load some useful math symbols/fonts
\usepackage{latexsym,amsfonts,amsmath,amssymb}

% comfort package to easily set margins
\usepackage[top=1in, bottom=1in, left=1in, right=1in]{geometry}
\usepackage{hyperref}
\usepackage[all]{hypcap}
% control some spacings
%
% spacing after a paragraph\begin{figure}[bth]
\setlength{\parskip}{.15cm}
% indentation at the top of a new paragraph
\setlength{\parindent}{0.0cm}

\begin{document}

\begin{center}
\Large
Ay190 -- Worksheet 11\\
David Vartanyan\\
Date: \today
\end{center}

\section{}

The moving Gaussian, upwind scheme, and FTCS are implemented in ws11.py. See Figures  ~\ref{fig:1} for upwind errors with $\sigma=\sqrt{15}$ and $\sigma=\sqrt{15}/5$, respectively.

Upwind is stable for all time if $\alpha = v \Delta t/\Delta x \leqslant 1$. 
\section{}
For FTCS, we see instability develop at late times as we increases our ntmax and thus our duration. See Fig~\ref{fig:2} with durations 200, 400s respectively. FTCS becomes unstable at $t \approx 150s$ unconditionally.

\section{}
The Lax-Friedrich method is an adaption of FTCS which has been made stable for $alpha \leqslant 1$ by adding a damping term to FTCS, resulting in poorer accuracy but stability. See Fig ~\ref{fig:3}. This method is less accurate than Upwind.

\section{}
Code is included for both Leapfrog and Lax-Wendroff.
For the Lax-Wendroff method, we see in Fig ~\ref{fig:4} that error scales as the square of the resolution so the method is indeed 2nd order.

However, this scaling drops off at later t. I couldn't identify why error diverges.

The reader may run all movies by removing hashes. 

\begin{figure}[bth]
\centering
\includegraphics[width=0.7\textwidth]{upwind1.png}
\caption{Upwind Error vs Time, $\sigma=\sqrt{15}$}
\includegraphics[width=0.7\textwidth]{upwind2.png}
\caption{Upwind Error vs Time, $\sigma=\sqrt{15}/5$}
\label{fig:1}
\end{figure}

\begin{figure}[bth]
\centering
\includegraphics[width=0.7\textwidth]{ftcs1.png}
\caption{Upwind Error vs Time, $ntmax=200$}
\includegraphics[width=0.7\textwidth]{ftcs2.png}
\caption{FTCS Error vs Time, $ntmax=400$}
\label{fig:2}
\end{figure}


\begin{figure}[bth]
\centering
\includegraphics[width=0.7\textwidth]{latfried.png}
\caption{LaxFried Error vs Time, $ntmax=100$, $\sigma=\sqrt{15}$}
\includegraphics[width=0.7\textwidth]{laxfried2.png}
\caption{Laxfried Error vs Time, $ntmax=100$, $\sigma=\sqrt{15}/5$}
\label{fig:3}
\end{figure}

\begin{figure}[bth]
\centering
\includegraphics[width=0.7\textwidth]{laxwend1.png}
\caption{Laxwend Error vs Time, $cfl=.5$}
\includegraphics[width=0.7\textwidth]{laxwend2.png}
\caption{Laxwend Error vs Time, $cfl=.25$}
\label{fig:4}
\end{figure}

\end{document}
