\documentclass[11pt,letterpaper]{article}

% Load some basic packages that are useful to have
% and that should be part of any LaTeX installation.
%
% be able to include figures
\usepackage{graphicx}
% get nice colors
\usepackage{xcolor}

% change default font to Palatino (looks nicer!)
\usepackage[latin1]{inputenc}
\usepackage{mathpazo}
\usepackage[T1]{fontenc}
% load some useful math symbols/fonts
\usepackage{latexsym,amsfonts,amsmath,amssymb}

% comfort package to easily set margins
\usepackage[top=1in, bottom=1in, left=1in, right=1in]{geometry}
\usepackage{hyperref}
\usepackage[all]{hypcap}
% control some spacings
%
% spacing after a paragraph\begin{figure}[bth]
\setlength{\parskip}{.15cm}
% indentation at the top of a new paragraph
\setlength{\parindent}{0.0cm}

\begin{document}

\begin{center}
\Large
Ay190 -- Worksheet 13\\
David Vartanyan\\
Date: \today
\end{center}

\section{}
Skeletal code is loaded and ready to go!

\section{}
My RHS  is an $m \times n$ matrix. The first three columns correspond to coordinate velocities, the last three coordinate accelerations.

I use RK2 integration to solve for the updated coordinate positions and velocities.

\section{}
The energy of the Earth-Sun gradually increases (keeping in mind negative convention), indicating the system is spiraling outwards. See ~\ref{fig:1}.

No difference is observed in changing resolution from $5000$ Nsteps to $10000$.

\section{}
We simulate sgrAstar for 100 years. The total energy evolves much worse than the Earth-sun system as we can see by comparing to the actual trajectory from UChicago Astro And UCLA astro. In fact, the energy becomes positive after a thresholdof a few years.

\begin{figure}[bth]
\centering
\includegraphics[width=0.7\textwidth]{ws131.png}
\caption{Earth-Sun Energy Evolution}
\label{fig:1}
\end{figure}

\end{document}
