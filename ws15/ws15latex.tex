\documentclass[11pt,letterpaper]{article}

% Load some basic packages that are useful to have
% and that should be part of any LaTeX installation.
%
% be able to include figures
\usepackage{graphicx}
% get nice colors
\usepackage{xcolor}

% change default font to Palatino (looks nicer!)
\usepackage[latin1]{inputenc}
\usepackage{mathpazo}
\usepackage[T1]{fontenc}
% load some useful math symbols/fonts
\usepackage{latexsym,amsfonts,amsmath,amssymb}

% comfort package to easily set margins
\usepackage[top=1in, bottom=1in, left=1in, right=1in]{geometry}
\usepackage{hyperref}
\usepackage[all]{hypcap}
% control some spacings
%
% spacing after a paragraph\begin{figure}[bth]
\setlength{\parskip}{.15cm}
% indentation at the top of a new paragraph
\setlength{\parindent}{0.0cm}

\begin{document}

\begin{center}
\Large
Ay190 -- Worksheet 15\\
David Vartanyan\\
Date: \today
\end{center}

\section{}

We solve the shock tube problem using smoothed particle hydrodynamics. We apply a particle treatment but average over density and pressure gradients by means of a smoothing kernel.

We introduce an artificial viscosity term to allow for nonadiabatic interactions (which would increase entropy), but we neglect self-gravity. 

The code is simple enough to fill out following the notes. It is also incredibly slow. In the plots below, the x-axis is distance and the y-axis is density.

You can see in Figures ~\ref{fig:1}, ~\ref{fig:2}, and ~\ref{fig:3} as we increase time material spills over shock front.

\section{}
 I did not have access to a Fortran compiler. 

\begin{figure}[bth]
\centering
\includegraphics[width=0.7\textwidth]{ws155.png}
\caption{Shock $0.0063$ seconds in}
\label{fig:1}
\end{figure}

\begin{figure}[bth]
\centering
\includegraphics[width=0.7\textwidth]{ws1525.png}
\caption{Shock $0.0315$ seconds in}
\label{fig:2}
\end{figure}

\begin{figure}[bth]
\centering
\includegraphics[width=0.7\textwidth]{ws15160.png}
\caption{Shocks $0.2$ seconds in}
\label{fig:3}
\end{figure}

\end{document}




